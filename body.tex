\section{VeloxDFS}
\subsection{Overview}
Contrary to what to a convetional description would, I would like to introduce
VeloxDFS by deepen into Apache Hadoop.

As we have discussed in the previous sections, one of the most important Big
Data processing framework is Apache Hadoop.  At the very hight level Hadoop as
two components: Hadoop File System and Hadoop MapReduce (Along YARN and other
essential services).

\\ \\ 

VeloxDFS is the component of another framework named Velox Big Data Framework
which similarly to Apache Hadoop has the same two main levels: VeloxMR,
MapReduce engine; and VeloxDFS, Distributed File System .  Nonetheless,
VeloxDFS can be also use with Apache Hadoop (Substituting HDFS), and not only
that, very interestly, later versions of VeloxDFS are solely compatible with
Hadoop for many reasons outside of the scope of this thesis.

\\ \\ 

One key difference with Hadoop file system that the reader already know is
about the elastic blocks, thus I would not deepen into this topic in this
section. Other key differences are regarding the topology of the network, while
Hadoop uses a standalone centralized network with a Namenode (and a qoura with
HA enabled), in  VeloxDFS each node is also a Namenode for a small subsets of
its files and blocks. We achieve this by representing the nodes and the files
in Distributed Hash table or \textit{DHT} using CHORD-like protocol which ensures
safe node entering and joining while also statistically splitting each of the
file metadata evenly accross the cluster. Also, similarly to HDFS we provide 
data recovery by means of data redundancy using replicas stored in different nodes.

\\ \\

As for technical aspects the whole file systems is written in C++ for being
able to access Operating System low-level components and it network stack is
based on BOOST ASIO using proactor model.

\subsection{Goals} 

As in many other parallel problems MapReduce systems can be quantified by a set of 
variables which describes the system how well does the system perfom versus a different one. 
In this work we focus in two key aspects: \textit{Job Execution time} and \textit{Load balancing}.
%three fundamental aspects: Job Throughput, Load balance, and Average job
%Latency. 
%
% \begin{itemize}
%     \item \textbf{Throughput} How much data is the system processing in a given time.
%     \item \textbf{Load balance} How well is the system diving the processing among its different components (Cores and machines). In this category, in MapReduce systems we have two main tenets:
%     \begin{enumerate}
%         \item \textbf{Inter-job load balancing}, how well the tasks within a job uses all available resources
%         \item \textbf{Between-jobs load balancing}, how well the jobs cooperate each other to extract all available resources of the system.
%     \end{enumerate}
%     \item \textbf{Latency} How fast can the system return the output of a job.
% \end{itemize}
 
\subsection{History}
Elastic blocks enabled VeloxDFS is the last iteration of many iterations of the Velox Big Data framework. The development of this framework has taken several years and it started from the study of distributed in-memory caches circa 2011 (cite Youngmoon and Deukyeon). Such study enlight us with the ideas of writing a mapreduce framework which benefits from inmemory distributed cache by using a cache aware scheduler. With that in mind we implemented EclipseMR in 2014, a MapReduce framework prototype which uses inmemory cache aware job scheduler and a consiting hashing underline distributed file system with also an alignment to the the inmmory cache. This design showed a greater performance compared to Hadoop and Spark in many of the single and multiple jobs standard bencharks (such as Terasort, WordCount, and Kmeans). EclipseMR was published at the IEEE Cluster 2017 at Honolulu, HW\\\\

With all these innovations in our project, by 2015 we decided to go a step forward and create an industry capable Big Data Framework utilizing all the novel techniques used in our previous research and prototypes. The codename for such framework was Velox (From the Latin Fast). \\
Velox requirements were: Double-ring Distributed hash table shaped inmemory cache and distributed filesystem; easily exentible design; Programable client API in Python, Java and C++; and most importantly iterative and DAG MapReduce jobs support. \\

Having all of those features in our backlog, I was assigned to design the system as a whole and started the implementation in early 2015 with few more colleagues. Concurrently, we enrolled an startup incubator program in where we try to gain traction to our OSS Velox Big Data Framework and potential find ways to monetize our work similarly to many other Apache fundation Big Data projects. \\
During that year most of our work was focus on finishing several milestones to keep up with our funding budget and traction. \\

Unfortunatelly, while we implemented most of our backlog, due to multiple technical issues the final performance was very poor compared to our rival systems (Such as Hadoop and Spark). Our man power to implement this system was very limited, undegraduate and graduate students who can participate in the code in their spare time. For all those reason, we gave up the idea of finding monetary ways to our projects and moved towards narrowing the scope of our requirements at the end of 2016. \\

We noticed that while the MapReduce engine was very promising, the Distributed file system was easier to tweek due to its simplicity. In the summer of 2017 explored the idea of enhancing current MapReduce frameworks throughput by using a custom underline Distributed file system which locates blocks at more convenient positions. This idea has an strong inspiration from previous works about reducing container initialization cost by coaslincing blocks in Hadoop. \\

For that purpose we moved an step forward and we decided to make VeloxDFS to generate logical block distributions consisting on the underline physical distributions. Those logical blocks can have arbitrary sizes, customized by a new logical block scheduler engine embebded in each FileLeader at the VeloxDFS. \\

After several iteration, we explored the idea about having arbitrary sized logical blocks in which their size changes dynamically at run-time (While a MapReduce job is running). Hence, for once we studied the idea of elastic blocks distributed file system. This work explores in detail about this idea and present what enhancement and challenges this idea brings.


\subsection{Architecture}
At the core, VeloxDFS is a decentralized userspace distributed file system written in C++. The file system relies on Chord-like Distributed Hash table, this is distributed consistent hashing to index files and blocks across the different nodes and to enable nodes leaving and entering the network with a safe manner\footnote{As for Novemver of 2018, the chord protocols for joining and exiting the network are to be implemented and it our backlog. Reasons are that so far we are still developing VeloxDFS and much of our efforts comes into finding novel ways to distribute the blocks}.


VeloxDFS instances are designed based on the pro-actor pattern, as result the server side ressembles an asynchronous RPC system. This was a key decision early in the developemnt which allows us to: avoid multithreading while providing conccurency; use our in-house network library (libvelox) while not locking VeloxDFS to it; and to separate business rules with networking issues. \\

\subsubsection{File Metadata}
An essential component of any file system is how to deal with its file metadata. To find effective ways to manage metadata we first need to understand how it is and how it is used in MapReduce systems. In MapReduce systems files are normally \textit{write once read many} WORM, this give us a hint of how metadata is frequently accessed but rarely written. Additionally, regarding its shape File Metadata is often much smaller than the data its refer.\\ 

Consequently, both of the property being accessed frequently and being small makes its a perfect candidate to be cached in memory and be easily indexed using our Chord like DHT Filesystem. To deal with the peculiarities of the metadata, we developed a complementely isolated component named FileLeader which implements all the business logic regarding metadata store and validation. 

\subsubsection{Blocks}
\lipsum[10]
\subsubsection{Network}
\lipsum[10]
\subsubsection{API}
\lipsum[10]

\subsubsection{Logical block schedulers}
\lipsum[10]

Lastly, As many other relatively complex systems, VeloxDFS its a composite of many small components that works at the unison, limited coverage would be given in this document as many of those components are trivial.


\subsection{Lean Scheduler}
The mechanism to assign the initial elastic blocks distribution and to control its resizing is done by the \textit{lean scheduler}. The scheduler is situated in both the client and the server side. In the server side the scheduler arranges its initial block distribitiouns explicitly at the fileleader making its best guesses using different techniques to construct logical blocks usings locally accesible phyical blocks. The client side of the lean scheduler is at the client side of VeloxDFS API to Hadoop (Invoke by Hadoop's \texttt{RecordReader}.

Due to the fact of having two types of schedules call, one initial and one (or more) at run-time, we need a way to partition the input data 
among the different scheduler calls. At the highest level, We split the input data into two segments: one for the initial block assignment and the remaining input data for the consequently run-time elastic block adjustment. The initial block assignment percentage is noted as $\alpha$ in this description

\subsubsection{Server side lean scheduler}
The server side of the lean scheduler competence is to write the initial phyisical blocks to logical blocks mappings, e.g. the initial logical block distributions. This scheduler invokation is done at the target file's \texttt{FileLeader}. As explained in the previous section the server side of the lean scheduler will commit a certain percentage $\alpha$ of the input data during the initial phase 

\subsubsection{Client side lean scheduler}
The client side of the lean scheduler compentece is to dynamically adjust the intially given logical blocks. This adjustment takes places after $\alpha I$ has been processed. The run-time block adjustements is done by dynamically assigning physical blocks to the current logical blocks. This is, whenever a tasks finishes with a physical block it will then attempt to process another phyiscal block's replica. Since there are multiple replicas we can consider that the slot with the highest throuput is more likely to process one of the physical blocks's replicas. Also, having this redundancy creates the need to have some sort of synchronization technique in the form of lock. In our work, we use a distributed lock system with as much locks as phyiscal blocks, in the following section we will cover the overhead consideration of this design.


\subsubsection{Overhead considerations}
The design of the lean scheduler obviates a particular bottleneck located at the distributed lock system. Such distributed lock system must maintain as much locks as physical chunks in the file that 
we are currently processing. Additionally, each of the tasks would concurrently attempt to lock the locks of its asigned physical chunks. This can be a problem when we scale our cluster to have more then 10000 slots, with each slots having hundres of physical chunks. Several approaches can be determined such as partitioning the locks accross multiple nodes, and buffering the lock request a transaction of multiple requests.  

\subsection{One-shot schedulers}
Our earlier approaches where based on the idea that we can only generate an static block distribution which can not be changed while the job is running.  The reason was solely based on the implementation issues, our initial Hadoop API consisted in a FileSystem API so that Hadoop would internally call Velox during the job in the same manner as it interacts with HDFS. The main drawback with this approach is that by default Hadoop's jobs only ask once at the beginning of the jobs regarding the block locations and its sizes. This seriously limitted our degree of action and restricted our logical scheduler to only have one chance of generating a logical block distribution. \\

Having in mind this strong limation, we first explored the idea of monitoring the system workload in each of the servers and generate a block distibution countering this system workload and later we explored the idea of generating logical block distributions that propiciate recursivly smaller map waves .


\subsubsection{IO aware block scheduler}
The first block scheduler that we considered creates logical block distributions upon of the current IO/CPU workload of the given cluster. For that purpose we collected Exponental Weighted Moving Averages (EWMA) of the current IO usage percentage read from UNIX tools such as \texttt{iostat} every an user specified amount of time. We also kept track of the load average of each machine to determine how many free cores each server has.

Upon this infromation the IO aware schedule will generate a etherable score of each of the replica of each phyiscal block. Such as score also consider the load balance in which the parameter to weight the load balance v.s. the IO score is $\alpha$.
For each physical block, its replica with the highest score would be assigned to its corresponding logical block. This resulted ina block distribution which mimincs the current system workload. 

While, this technique resulted in good performance some times. We found that often the system workload could rapidly change, thus rendering our well crafted block distribution outdated.

\subsubsection{Recursive block scheduler}
One of the main drawbacks of reling in workload heuristics is that the system workload is ever changing, this is as much as we generate block distribution which mimics the system workload at a given time, this is only valid for a short period after the measurement was taken, this is, much shorter than what a normal job would take. This is a very fundamental limitation for the IO aware scheduler which forced us to give up that approach and become more creative in our quest to design an schedule which create fine logical block distributions. \\

A breakthrough came into the form of realizing that most of our obvserved MapReduce jobs's tasks ending time has an approximately 10s variance. This inspired the creation of a the recursive block scheduler which creates logical blocks 

This is, it only takes a straggling task to delay the whole job execution time. The approach to fix this was to generate a logical block distribution consisting in initial large logical blocks followed by one or more waves of recursivly smaller logical blocks. By doing that we hoped Hadoop to start scheduling those large blocks first and start allocating the smaller blocks in the slots which are free (this is, the good performing slots). \\
This idea gave good load balance results, however, often Hadoop would not honor our request to schedule first the large logical blocks saboteting our idea and finally rendering our idea useless.

\subsection{Technical considerations}
\lipsum[10]
